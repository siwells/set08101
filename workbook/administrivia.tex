\chapter{Administrivia}

\section{Introduction}
\label{intro}

\paragraph{} Welcome to the Web Technologies module from Edinburgh Napier University. This module has a slightly different structure to many modules so it's worth reading through this guidance before you get stuck into the good stuff.

%\paragraph{} This module is about the Web, what the Web is, what we can currently do with the Web, and what the Web might be like in the future. Rather than focus on the user side of web technology, such as CSS and Javascript, which we've all probably seen in other modules (particularly \emph{web tech}, the pre-requisite for this module), we are going to take a more holistic approach and examine both the server and the client side. We shall look at HTTP (the core of all web technologies) and learn how APIs, Web Services, \& RESTful architectures are built to move data around. We will then delve into some more topical discussions, starting with security \& privacy on the web, then looking at adding intelligence (Semantic Web), adding increased, scalable interaction (Realtime Web) and private, anonymous, and un-censorable web-technologies (Dark Web). We wrap up the module by examining a few technologies that might form the basis of future web capabilities like Blockchain and IPFS. Throughout the module you should be using the topics as a starting point. A starting point for your own, self-directed, exploration of the topic in more depth. Because we can only really survey any given aspect during contact time; the real knowledge comes from you digging into it in depth. Also a starting point for you to critically reflect on your own experiences and responses to the topics and issues that we raise in class. Whilst the Web itself is a technical mechanism, a tool for moving data around, it also operates in a very complex socio-technical context that currently affects, and will likely affect to an even larger degree in the future, all our lives.

%\paragraph{} Lectures and labs will not necessarily align neatly, the labs will progress, on a week-by-week and chapter-by-chapter basis to form a first course in ``\emph{developing server side web-apps using Python and associated professional practises in a Linux-based environment}''. The lectures will summarise some of this material, providing an opportunity to talk over what we learned in the lab, but primarily, the lectures are an opportunity for us to discuss many other aspects of the Web, focussing in particular on two aspects. Firstly, the structure of the existing Web. Secondly the various facets of Web technologies that influence or alter the way that we use, manipulate, navigate, and share information. These break down into a series of named facets: the semantic web, the realtime web, the dark web, and the permanent web, each of which imposes its own constraints on and proffers opportunities for what the Web is and can do. By looking at each of these in turn we should gain some insight into how the Web is developing, the directions it might take in the future, and perhaps, suggest areas that we can usefully and positively affect.

\paragraph{} How should we use this workbook? Ideally we would work through it on a chapter-by-chapter basis supplementing our work with background reading and wider exploration of each topic introducted. Some chapters will take longer to complete than others, and other chapters will need to be returned to multiple times. This is particularly true for the first two chapters. To learn both Linux and Python in a fortnight is a tall order so I'd suggest working iteratively, do enough to make some progress, then frequently return to the respective chapters to learn a little more, usually by following the links and footnotes to further practise materials.

\paragraph{} In the first week work through the first chapter in the lab section of the workbook. You can read ahead if you want but don't try to run any Flask web-apps on the dev server until you've been assigned your personal virtual server to run your own web-apps on. This week is mostly concerned with the foundation of our learning environment. Logging in, learning to navigate and do simple tasks at the command line, using a non-gui text editor, and using Git. There are links, usually in footnotes, throughout the chapter, for example, to practise the Linux command line then there are online web sites like the \emph{LinuxZoo}\footnote{\url{http://www.linuxzoo.net}} that you can use to practise your skills. Similarly, the links to Vim practise tutorials, particularly the Vim game, will help you practise the skills you need to work efficiently in subsequent weeks. Finally, make sure to work through the linked Git tutorials and ensure that you are confident that you understand each of these tools and their place within the learning environment before moving on to subsequent chapters. 

\paragraph{} Each chapter is meant to cover about an entire week of study, so don't rush through things within the scheduled lab session just to tell yourself the you've done all the work. As I mentioned in the introduction, topics, whether in lectures or labs are meant to be a starting place, a framework to guide your self-directed study, but not the totality of your learning. 

\paragraph{} In subsequent weeks you will start to build knowledge of the Python language and the Flask library which provides functionality for building server-hosted web-apps. The next chapter, on Python, is meant to cover at least a weeks work and deliberate practise. Mostly that week is concerned with developing basic skills in a new language, Python, which is actually quite a challenge. This is not because Python is particularly difficult but because learning a computer language well takes time and effort and you have to start somewhere. Subsequent weeks will require you to work through various chapters of the workbook. You will find that as you progress you will want to skip ahead to different sections, especially once the assessments are released and you want to include specific functionality in your coursework. So after about chapter 3 I expect that many of you will navigate your own path through the remainder of the workbook. The only proviso is that you should aim to have worked through every chapter by the end of the module.


\section{Reading This Book}
\label{readme}
\paragraph{} This book covers a variety of technologies, some of which you might already be familiar with. Also, depending upon your goals, you might have specific topics that you want to study. We'll cover the following topics:

\begin{enumerate}
\item The World Wide Web - An Introduction \& Overview 
\item Hypertext+Markup+Information+Semantics=HTML 
\item CSS Intro \& Overview
\item HTML Page Layout Using CSS
\item Design for Hackers
\item Core JS
\item Client-Side JS: Browser APIs
\item Client-Side JS: Data Storage APIs 
\item Client-Side JS: Sound \& Vision APIs 
\item Deployment
\end{enumerate}

\paragraph{} THe following figure suggests a few routes through our topics, grouping together thematically related topics, and ensuring that pre-requisit chapters are indicated.

\begin{figure}[H]
\centering
\includegraphics[width=0.8\textwidth]{figures/reading-order.pdf}
\label{fig:reading-order}
\end{figure}


